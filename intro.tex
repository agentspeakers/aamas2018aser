\section{Introduction}
\label{sec:intro}


%
% About AgentSpeak(L)
%
{\asl} has been introduced in \cite{Rao96}  with the purpose of defining an expressive, 
abstract language capturing the main aspects of the Belief-Desire-Intention architecture~\cite{Bratman88,Georgeff:1987:RRP:1863766.1863818}, featuring a formally defined semantics and an abstract interpreter.
%
The starting point to define the language were real-world implemented systems, namely the
Procedural Reasoning System (PRS)~\cite{Ingrand:1992:ARR:629535.629890} and the Distributed Multi-Agent Reasoning System (dMARS).
%

%
% Concrete APL
% 
{\asl}  and PRS have become a main reference for implementing concrete Agent Programming Languages 
based on the BDI model: 
%
main examples are Jason~\cite{jason06,bordini:07} and ASTRA~\cite{DBLP:conf/prima/CollierRL15}.
%
Besides Agent Programming Languages,  the {\asl}  model has been adopted as the main reference to development several BDI agent-based frameworks and technologies~\cite{BordiniMAPlpa,BordiniMAPlta}.


%
% Contribution
%
Existing Agent Programming Languages extended the language with constructs and mechanisms making it  practical from a programming point of view~\cite{jason06}.
%
Besides, proposals in literature extended the model is order to make it effective for specific kind of systems---e.g., real-time systems~\cite{Vikhorev:2011:APP:2030470.2030529}, XXX --- or to improve the structure of programs, e.g. in terms of modularity~\cite{Madden2010,Nunes2014}.

%
Along this line, in this paper we describe a novel extension of the {\asl}  model -- that we called {\aser} -- featuring \emph{plan encapsulation}, i.e. the possibility to define plans that fully encapsulate the strategy to achieve the corresponding goals, integrating both the pro-active and the re-active behaviour.
%
% Key points
%
This extension turns out to bring a number of important benefits to agent programming based on the BDI, namely:
%
\begin{itemize}
\item encapsulation of goal-oriented behaviour -- ...
\item modularity of the agent code -- ...
\item failure handling -- ...
\item runtime management of goals and intentions -- ...
\end{itemize}
%
\noindent Besides the benefits in terms of agent programming, the approach:
%
\begin{itemize}
\item reduces the gap between the design level and the programming level,  ...
\item ease goal-based reasoning -- ...
\end{itemize}

The remainder of the paper is organised as follows:
%
first we describe in details the motivation that lead to the proposal (Section \ref{sec:motivation});
%
then, we introduce and discuss {\aser},  introducing the main concepts, syntax and semantics---first informally  (Section \ref{sec:proposal}) and then providing the formalisation of some key aspect (Section \ref{sec:formalisation}).
%
Finally we discuss the results of a first evaluation that has been carried on, based on a prototype implementation extending the ASTRA platform (Section \ref{sec:evaluation}).
%


.