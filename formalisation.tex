\section{Semantics}
\label{sec:formalisation}

%% First update reasoning cycle states...

Briefly, these are the changes in the semantics of AgentSpeak(L)
that are needed to accomodate the new features of AgentSpeaker(ER):

\begin{itemize}
\item in the beginning of the reasoning cycle, check all
  goal-conditions starting from the bottom of each intention stack and
  remove all finished intentions;

\item if an event is \emph{external}, try to trigger one
  relevant/applicable plan for each intention stack, starting from the
  top of the intention stack (that is, the most specific plan in the
  g-plan tree);

\item if an event is \emph{internal}, search for a relevant applicable
  plan on the path to the root of the g-plan tree, that is, starting
  from the most specific plans in the g-plan tree (as in the item
  above but now there is only one intention to refer to); this can be
  done using prefixes to goal names and does not change the semantics
  per se, only the order in which we check for relevant applicable
  plans.
\end{itemize}

We start by updating the well-known reasoning cycle~\cite{JasonBook}
to adapt it for the initial stage of checking all goal-conditions to remove
the intentions that should no longer be active. Recall that this can
be because the goal has been achieved, has become impossible to
achieve, or the motivation why it was adopted no longer
applies. Furthermore, note that even though this is computationally
costly, it is necessary because not doing it at every reasoning cycle
may imply the agent missing the moment where the goal-condition became
true (hence the g-plan needing to be deactivated); it is also worth
the computational burden in as much as it has the various practical
programming advantages we pointed out earlier in this paper.

The required stage for checking goal-conditions is included in the
existing $\ClrInt$ stage (which previously only removed empty
intentions) except it is now moved to the beginning of the reasoning
cycle (to ensure nothing in the reasoning cycle is done under the
assumption a deactivated goal is still being pursued), just after the
$\ProcMsg$ stage (as the information just received from other agents
might be useful in checking for goals to be deactivated). The clearing of
intentions used to be the last part of the reasoning cycle, but
because there are no other dependencies between the first and last
stages, $\ClrInt$ might as well be done at the beginning rather than
the end. The slightly changed reasoning cycle is shown in
Figure~\ref{fig:rcaser}.

\begin{figure}[htbp]
  \begin{center}
    \includegraphics[width=\linewidth]{ASERrc.pdf}
    \caption{The AgentSpeak(ER) Reasoning Cycle}
    \label{fig:rcaser}
  \end{center}
\end{figure}

The \ClrInt stage also needs a new semantic rule. To facilitate the
presentation of that rule, we define a new auxiliary function
$\GCOND$, which given a g-plan simply returns the goal-condition
component of that plan, as follows. Let
$p="\mathtt{+!g : b <: c <- \{ a.\}}"$, then $\GCOND(p)=c$, more
specifically the logical formula coded by $c$. Also, we denote by
$[p_1,...,p_n]$ an intention stack with plan $p_n$ at its top.

The new rule \rn{ClrInt$_4$}, as shown below, is added to the 3
existing ones (see~\cite[p~212]{JasonBook}). It essentially removes
from intentions the bottom most g-plan for which ist goal-condition
now follows from the state of the belief base. The intuition is that
if goal $\texttt{!g1}$ required a subgoal $\mathtt{!g2}$ (a plan for
which was pushed on top of the plan for the former one in the
intention stack) and $\texttt{!g1}$ is no longer active, that whole
part of the intention stack above it needs to be removed together with
it.

\infrule[ClrInt$_4$] {
        i \in \CI \qquad i=[p_1,\dots,p_n] \\
        \AGBELS \models \GCOND(p_j) \mbox{ for some } j, 1\leq j\leq n
}
%------------------------------------------------
{   \CFG{\ClrInt}  \trans  \CFGcp{\ClrInt} \\[1.2mm]  
\begin{array}{llcl}
  \mbox{\emph{where:}\quad}
     & \multicolumn{3}{l}{\mbox{$j$ is the least number in [1..n]
       s.t.}}\\
     & & & \AGBELS \models \GCOND(p_j)\\
     & \CIli & = & \CI \setminus \{i\} \cup \{[p_1,\ldots,p_{j-1}]\} \\
\end{array}
}

The auxiliary functions $\RELPLANS(\PLANS,\TE)$
$\APPLPLANS(\PLANS,\TE)$ (see Definitions~10.2 and~10.3
in~\cite{JasonBook}) can be changed to accommodate both the new g-plan
structure as well as the firing of reaction plans (i.e., plans for
external events) for all intentions rather than creating a single new
separate intention as before. First, the redefined functions now
receive a set of intentions as an extra parameter; they are always
called with the current contents of the set of intentions ($\CI$ in
the operational semantics). If $\TE$ is an external event, the new
parameter is used to check for relevant/applicable plans for each
individual intention plus the empty intention $\EXT$. The empty
intention being included to the set of intentions in the parameter
ensures that relevant reaction plans for the ``main''
goal\footnote{Recall that this is a default g-plan that is implicitly
  associated with top-level plans, as in traditional AgentSpeak(L).}
will be checked; this in turn allows for backwards compatibility with
AgentSpeak(L). Besides having an extra parameter, the functions now
return a set of triples rather than a pair. We use $\TREV(p)$ to refer
to the triggering event of a plan $p$. Formally, we have:

\begin{definition}[Relevant Plans]
  The auxiliary function to retrieve relevant plans given a plan
  library $\PLANS$, a particular event $\TE$ of type \texttt{+l} or
  \texttt{-l} (i.e., an external event, one reacting to changes in beliefs
  rather than a goal adoption event), and a set of intentions $\DomI$
  is defined as
  $\RELPLANS(\PLANS,\TE,\DomI) = \{ \langle\PLANS,\theta,i\rangle
  \mid i \in \DomI, i = [\PL_1,\ldots,\PL_n],$
  and $j$ is the greatest number in $[1..n]$ s.t.\ $\PL_j$ is a g-plan
  with a (reaction) plan $rp$ in scope and
  $\{\TE\} \models \TREV(rp)\theta \}$, for some m.g.u.\ $\theta$. By
  ``in scope'' we mean that plan $rp$ appears within the g-plan for
  $\PL_j$ (as for example ??? in Figure~\ref{???}). For internal
  events the function returns a set containing a single element
  $\langle\PLANS,\theta,i\rangle$ where $i$ is the specific intention
  that generated $\TE$ and $\PLANS$ is the set of relevant plans using
  the same idea of scope as above for external events.
\end{definition}
%% RHB: fix reference to an example in another section here.

We do not formally define the $\APPLPLANS$ function here due to space,
and given that it is trivially extended in a very similar way as in
Definition~\ref{def:relplans} for $\RELPLANS$.

Finally, we need to change the semantics so that not just one but
\emph{all} relevant reaction plans are triggered, that is, one for
each active intention, and within a single intention starting from the
most specific plan (i.e., the one closest to the top of the intention
if there are relevant plans at other levels too). Most of the work was
already done in the redefinitions of the $\RELPLANS$ and $\APPLPLANS$
functions, but we still need to change the rule for processing
external events (which now change multiple intentions). Note however
that by the time we come to handle an external event with rule
\rn{ExtEv}, a single plan for each intention has already been
selected; that is, $\SO$ (the option selection function) is also
slightly redefined to work with the new structures returned by those
redefined auxiliary functions.

The new \rn{ExtEv} rule is below, and it essentially says that an
external event now potentially interrupts various intentions, provided
there are relevant and applicable plans anywhere in the g-plans
associated with each particular current intention of the agent. Recall
that external events for the ``main'' goal --- the external events as
in traditional AgentSpeak --- will now be associated with the empty
intention \EXT in \TRHO (i.e., the result of the $\RELPLANS$ and
$\APPLPLANS$ functions). The chosen plan, after being applied its
respective variable substitution $\theta$, for each intention is
pushed on top of that intention (the notation used for this below is
$i_j[\PL_j\theta_j]$).

\infrule[ExtEv] {
        \TEPS = \langle\TE,\EXT\rangle \qquad \TRHO =
        \{\langle\PL_1,\theta_1,i_1\rangle,\ldots,\langle\PL_n,\theta_n,i_n\rangle\}
}
%------------------------------------------------
{   \CFG{\ClrInt}  \trans  \CFGcp{\ClrInt} \\[1.2mm]  
\begin{array}{llcl}
  \mbox{\emph{where:}\quad}
     & \CIli & = & \CI \setminus \bigcup_{j=1}^{n} \{i_j\} \cup
                   \bigcup_{j=1}^{n} \{i_j[\PL_j\theta_j] \}\\
\end{array}
}

To conclude this section, we emphasise that, for the sake of space, we
here formalised only the main changes to the well-known AgentSpeak
semantics. The complete new transition system giving semantics to
AgentSpeak(ER) should appear in a much longer paper.
