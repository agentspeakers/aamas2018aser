%%
%% sample document for AAMAS'18 conference
%%
%% modified from sample-sigconf.tex
%%
%% see ACM instructions acmguide.pdf
%%
%% AAMAS-specific questions? n.yorke-smith@tudelft.nl
%%

\documentclass[sigconf]{aamas}  % do not change this line!

%% your usepackages here, for example:
\usepackage{booktabs}
\usepackage{xcolor}
 \usepackage{url}
\usepackage{color}
\usepackage{listings}
\usepackage{flushend}



%% do not change the following lines
\setcopyright{ifaamas}  % do not change this line!
\acmDOI{}  % do not change this line!
\acmISBN{}  % do not change this line!
\acmConference[AAMAS'18]{Proc.\@ of the 17th International Conference on Autonomous Agents and Multiagent Systems (AAMAS 2018)}{July 10--15, 2018}{Stockholm, Sweden}{M.~Dastani, G.~Sukthankar, E.~Andr\'{e}, S.~Koenig (eds.)}  % do not change this line!
\acmYear{2018}  % do not change this line!
\copyrightyear{2018}  % do not change this line!
\acmPrice{}  % do not change this line!

%% the rest of your preamble here

\newcommand{\code}[1]{\texttt{#1}}

\newcommand{\aser}{\textsf{AgentSpeak(ER)}}
\newcommand{\jason}{\textsf{Jason}}
\newcommand{\AandA}{\textsf{A\&A}}
\newcommand{\aanda}{\textsf{A\&A}}
\newcommand{\moise}{\textsf{MOISE}}
\newcommand{\cartago}{\textsf{CArtAgO}}
\newcommand{\jacamo}{\textsf{JaCaMo}}



%%%%%%%%%%%%%%%%%%%%%%%%%%%%%%%%%%%%%%%%%%%%%%%%%%%%%%%%%%%%%%%%%%%%%
\newcommand{\commenttoby}[3]{\sm{#1}{#2 \textbf{--#3}}}
\newcommand{\ste}[2]{\sm{#1}{#2 \textbf{--Ste}}}
\newcommand{\FORGET}[1]{}
\newcommand{\tba}[1]{\color{blue} TO BE ADDED: #1 \color{black}}
\newcommand{\ale}[1]{\color{blue} [ALE]: #1 \color{black}}
\newcommand{\more}{\color{blue} [...] \color{black}}
\newcommand{\tbc}[1]{\color{red} TO BE CLARIFIED: #1 \color{black}}
% \newcommand{\todo}[1]{\color{red} TODO: #1 \color{black}}
\newcommand{\tbd}[1]{\color{orange} TO BE DISCUSSED: #1 \color{black}}
\newcommand{\FIXME}[1]{{{\bf \color{red}{#1}}}}
%%%%%%%%%%%%%%%%%%%%%%%%%%%%%%%%%%%%%%%%%%%%%%%%%%%%%%%%%%%%%%%%%%%%%%


\sloppy
%%%%%%%%%%%%%%%%%%%%%%%%%%%%%%%%%%%%%%%%%%%%%%%%%%%%%%%%%%%%%%%%%%%%%%%%%%%%%%%%%%%%%%%%%%%%%%%%%%%%%%%%%

\begin{document}

\title{AgentSpeak(ER): An Extension of AgentSpeak(L) improving
  Encapsulation and Reasoning about Goals}

% put your title here!
%\titlenote{Produces the permission block, and copyright information}

% AAMAS: as appropriate, uncomment one subtitle line; check the CFP
\subtitle{Extended Abstract}
%\subtitle{Industrial Applications Track}
%\subtitle{Socially Interactive Agents Track}
%\subtitle{Blue Sky Ideas Track}
%\subtitle{Robotics Track}
%\subtitle{JAAMAS Track}
%\subtitle{Doctoral Mentoring Program}

%\subtitlenote{The full version of the author's guide is available as \texttt{acmart.pdf} document}


% AAMAS: submissions are anonymous for most tracks
% \author{Paper \#329}  % put your paper number here!

%% example of author block for camera ready version of accepted papers: don't use for anonymous submissions
%
\author{Alessandro Ricci}
%\orcid{1234-5678-9012}
\affiliation{%
  \institution{DISI, University of Bologna}
  \streetaddress{Via Sacchi, 3}
  \city{Cesena} 
  \state{Italy} 
  % \postcode{47521}
}
\email{a.ricci@unibo.it}
%
\author{Rafael H. Bordini}
%\authornote{The secretary disavows any knowledge of this author's actions.}
\affiliation{%
  \institution{FACIN-PUCRS}
  \streetaddress{Av. Ipiranga, 6681}
  \city{Porto Alegre, RS} 
  % \postcode{90619-900}
  \state{Brazil} 
}
\email{r.bordini@pucrs.br}
%
\author{Jomi F. H\"ubner}
% \authornote{This author is the
%  one who did all the really hard work.}
\affiliation{%
  \institution{DAS, Federal University of Santa Catarina}
  \streetaddress{PO Box 476}
  \city{Florian�polis, SC} 
  %\postcode{88040-900}
  \country{Brasil}
}
\email{jomi.hubner@ufsc.br}
%
\author{Rem Collier}
\affiliation{%
  \institution{University College of Dublin}
  \city{Dublin}
  \country{Ireland}
}
\email{rem.collier@ucd.ie}

\begin{abstract}  
  In this paper we introduce AgentSpeak(ER), an extension of the
  AgentSpeak(L) language tailored to support encapsulation. The
  AgentSpeak(ER) extension aims at improving the
  style of BDI agent programming along relevant aspects, including
  program modularity and readability, failure handling, and reactive
  as well as goal-based reasoning.
\end{abstract}

\begin{CCSXML}
<ccs2012>
<concept>
<concept_id>10010147.10010178.10010219.10010221</concept_id>
<concept_desc>Computing methodologies~Intelligent agents</concept_desc>
<concept_significance>500</concept_significance>
</concept>
</ccs2012>
\end{CCSXML}

\begin{CCSXML}
<ccs2012>
<concept>
<concept_id>10011007.10011006.10011050</concept_id>
<concept_desc>Software and its engineering~Context specific languages</concept_desc>
<concept_significance>500</concept_significance>
</concept>
</ccs2012>
\end{CCSXML}

\ccsdesc[500]{Computing methodologies~Intelligent agents}
\ccsdesc[500]{Software and its engineering~Context specific languages}

% AAMAS: the ACM CCS are not needed within AAMAS papers
%%
%% The code below should be generated by the tool at
%% http://dl.acm.org/ccs.cfm
%% Please copy and paste the code instead of the example below. 
%%
%\begin{CCSXML}
%<ccs2012>
% <concept>
%  <concept_id>10010520.10010553.10010562</concept_id>
%  <concept_desc>Computer systems organization~Embedded systems</concept_desc>
%  <concept_significance>500</concept_significance>
% </concept>
% <concept>
%  <concept_id>10010520.10010575.10010755</concept_id>
%  <concept_desc>Computer systems organization~Redundancy</concept_desc>
%  <concept_significance>300</concept_significance>
% </concept>
% <concept>
%  <concept_id>10010520.10010553.10010554</concept_id>
%  <concept_desc>Computer systems organization~Robotics</concept_desc>
%  <concept_significance>100</concept_significance>
% </concept>
% <concept>
%  <concept_id>10003033.10003083.10003095</concept_id>
%  <concept_desc>Networks~Network reliability</concept_desc>
%  <concept_significance>100</concept_significance>
% </concept>
%</ccs2012>  
%\end{CCSXML}
%
%\ccsdesc[500]{Computer systems organization~Embedded systems}
%\ccsdesc[300]{Computer systems organization~Redundancy}
%\ccsdesc{Computer systems organization~Robotics}
%\ccsdesc[100]{Networks~Network reliability}


\keywords{Agent-Oriented Programming; Agent Programming Languages; BDI; AgentSpeak(L); Jason; ASTRA; AgentSpeak(ER)}  % put your semicolon-separated keywords here!

\maketitle


%%%%%%%%%%%%%%%%%%%%%%%%%%%%%%%%%%%%%%%%%%%%%%%%%%%%%%%%%%%%%%%%%%%%%%%%%%%%%%%%%%%%%%%%%%%%%%%%%%%%%%%%%
%% start of main body of paper

%%%%%%%%%%%%%%%%%%%%%%%%%%%%%%%%%%%%%%%%%%%%%%%%%%%%%%%%%%%%%%%%%%%%%%%%%%%%%%%%%%%%%%
\section{Introduction}
\label{sec:intro}
%%%%%%%%%%%%%%%%%%%%%%%%%%%%%%%%%%%%%%%%%%%%%%%%%%%%%%%%%%%%%%%%%%%%%%%%%%%%%%%%%%%%%%

%
% About AgentSpeak(L)
%
{\asl} has been introduced in \cite{Rao96}  with the purpose of defining an expressive, 
abstract language capturing the main aspects of the Belief-Desire-Intention architecture~\cite{Bratman88,Georgeff:1987:RRP:1863766.1863818}, featuring a formally defined semantics and an abstract interpreter.
%
The starting point to define the language were real-world implemented systems, e.g.,
Procedural Reasoning System (PRS)~\cite{Ingrand:1992:ARR:629535.629890}.
% and the Distributed Multi-Agent Reasoning System (dMARS).
%
%
% Contribution
%
Existing Agent Programming Languages extended the language with
constructs and mechanisms making it practical from a programming point
of view~\cite{jason06,DBLP:conf/prima/CollierRL15}.

%
Along this line, in this paper we describe a novel extension of the
{\asl} model --- called {\aser} --- featuring \emph{plan
  encapsulation}, i.e. the possibility to define plans that fully
encapsulate the strategy to achieve the corresponding goals,
integrating both the pro-active and the reactive behaviour.
%
% Key points
%
%

%%%%%%%%%%%%%%%%%%%%%%%%%%%%%%%%%%%%%%%%%%%%%%%%%%%%%%%%%%%%%%%%%%%%%%%%%%%%%%%%%%%%%%
\section{Motivation}
\label{sec:motivation}
%%%%%%%%%%%%%%%%%%%%%%%%%%%%%%%%%%%%%%%%%%%%%%%%%%%%%%%%%%%%%%%%%%%%%%%%%%%%%%%%%%%%%%

The main motivation behind {\aser} comes from the experience using
agent programming languages based on the {\asl} model, {\jason} and
ASTRA in particular.
%%
Yet, these issues are relevant for any language based on the BDI
architecture.
%%
%The first issue is about the plan model.

% \subsection{Plan encapsulation}
%
In the BDI model, plans are meant to specify the means by which an
agent should satisfy an end~\cite{Rao96}.
%
In {\asl} a plan consists of a rule of the kind \textsf{e : c <- b}.
%
The head of a plan consists of a triggering event \textsf{e} and a
context \textsf{c}.
%
The triggering event specifies why the plan was triggered, i.e., the
addition or deletion of a belief or goal.
%
In the following, we refer to plans triggered by event goals
as \emph{g-plans}, and plans triggered by the
belief change (including percepts) as \emph{e-plans}.
%
The context specifies those beliefs that should hold in the agent's
set of base beliefs if the plan is to be triggered.
%
The body of a plan is a sequence of actions or (sub-)goals.
%

%
% THE PROBLEM
%
In this approach --- as well as in planning, in general --- the
\emph{means} to achieve a goal --- i.e., the body --- is meant to be
fully specified in terms of the actions the agent should execute and
the (sub-)goals the agent should achieve or test.
%
In the programming practice, however, it is often the case that the
strategy (the means) adopted to achieve some goal (the end) would
naturally include also \emph{reactions}, i.e., reacting to events
asynchronously perceived from the environment, including changes about
the beliefs.
% 
That is, an agent is not reacting only to unexpected events so as to
change/adapt its course of actions, but reactivity could be an
effective ingredient of the strategy, of the plan adopted to achieve
some goal.
%
Besides the programming practice, this is often the case if we
consider human activities too. Reactivity is a key ingredient of many
activities that we perform to achieve specific goals, not only to
handle events that represents errors or unexpected situations (for the
current courses of actions).
%
It follows naturally that this is also an opportunity to extend the
plan model so as to fully \emph{encapsulate} also reactions that are
part of the strategy to achieve the goal, as well as the subgoals that
are specific to that particular goal.



%
Let us consider the robot cleaning example used to describe plans in
{\cite{Rao96}}.
%
One of the plans is:

\begin{small}
\begin{verbatim}
+location(waste,X):location(robot,X) & location(bin,Y)
  <- pick(waste); !location(robot,Y); drop(waste).
\end{verbatim}
\end{small}

\noindent That is, as soon as the robot perceives that there is a
waste in its location, then it can pick it up and bring it to the bin.
%
This plan an essential brick of the overall strategy to achieve the
goal of cleaning the environment --- or to maintain the environment
clean, in a \emph{maintenance goal} view.
%
However, the fact that this plan is useful to achieve that goal is not
explicit in the source code, it remains in the mind of the
programmer/designer.
%
The agent reacts to a waste in a location because it has an implicit
goal, which is about cleaning the environment.
%
However, since there are no g-plans about it, there is no explicit
trace in the agent mental structures about this goal.


%%%%%%%%%%%%%%%%%%%%%%%%%%%%%%%%%%%%%%%%%%%%%%%%%%%%%%%%%%%%%%%%%%%%%%%%%%%%%%%%%%%%%%
\section{A Taste of {\aser}}
\label{sec:proposal}
%%%%%%%%%%%%%%%%%%%%%%%%%%%%%%%%%%%%%%%%%%%%%%%%%%%%%%%%%%%%%%%%%%%%%%%%%%%%%%%%%%%%%%

{\aser} extends the plan model of {\asl}  beyond the simple sequence of
actions and goals, so as to \emph{(i)} include also the possibility to specify
reactive behaviour encapsulated within the plan, in terms of e-plans;  \emph{(ii)}  
 enforcing that each e-plan would be defined always in the scope
of some g-plan---i.e., reactions occur always in the context of some 
explicit designed goal to achieve.

The {\aser} syntax for plan definition extends the {\asl} as follows:

{\small
\begin{verbatim}

/*  g-plans to achieve goal g in context c */

+!goal : context <: goal_cond { 

  <- ... // main sequence (body actions)

  /* encapsulated e-plans */
  +e1 : c1 <- ...	
  +e2 : c2 <- ....
  
  /* encapsulated g-plans */ 
  +!g1 {    
    <- ...    
    +e3 : c3 <- ... // possible old-style plans
  }	

  /* encapsulated e-plans catching failures */
  -!g1 : ... <- ... catches from failures 
}
\end{verbatim}}

%
% \noindent Coherently, with the {\asl} model, such a behaviour can be expressed
% in terms of e-plans.
%
% JH: I changed a bit the terminology
\noindent A plan becomes the \emph{scope} of ($i$) a sequence of
actions (referred as \emph{body actions}), ($ii$) a set of
\emph{e-plans}, specifying a reactive behaviour which is active at
runtime only when the plan is in execution, and ($iii$) a set of
\emph{g-plans}, specifying plans to achieve subgoals that are relevant
only in the scope of this plan. The e-plans and g-plans are referred
to as \emph{sub-plans}.
%
The sub-plans may include also reactions to failures occurring when
the plan is executed.

The robot cleaning example becomes:
%
\begin{small}
\begin{verbatim}
+!clean_env  {
   +location(waste,X) : location(robot,X) & location(bin,Y)
      <- pick(waste); !location(robot,Y); drop(waste).
}
\end{verbatim}
\end{small}
%
\noindent We can give an explicit reason for the reactive behaviour by
encapsulating the e-plan inside a g-plan, with an explicit goal \texttt{clean\_env}.
%
This is also a particular case where the body of the g-plan does not
have any actions.

%
%% RHB: So the failure handling plans for the outer goal were moved
%% inside (as well as out)? I'm against this, but I guess this is a
%% democracy :D
%
Informal semantics of the extended plan model:
\begin{itemize}

\item The sub-plans are part of the strategy which can be applied to achieve the g-plan.
%
The main sequence (body actions) can be empty---this is typical of purely reactive behaviours.
%
If an event triggering a sub-plan is triggered while the g-plan is executing, i.e. the main sequence 
is in execution, and the sub-plan is applicable according to the context, then the body of the sub-plan 
is stacked on top of the stack of the current (g-plan related) intention.
%
The effect is like an asynchronous interruption of the main sequence, to execute the body of the sub-plan.

\item The execution of the extended plan is considered completed if/when the condition described by 
the \texttt{goal\_cond} expression is met.
%
%GoalCondition is analogous to the context: however, if context can be considered to be as the pre-condition to execute the plan, the GoalCondition are like a post-condition, which corresponds to a declarative description of the goal to be achieved. 
%
%Alternatively, the execution of a plan can be forced to complete by using an internal action .done - this is useful everytime the goal condition cannot be effectively or naturally described as a boolean expression.
%
If \texttt{goal\_cond} is not specified, by default the condition is that all actions of the main sequence have been executed and completed, in continuity with the AS model, unless the case in which the main plan body is empty: in that case the condition is {\texttt{false}} by default.
%
%, i.e. the plan This condition is explicitly represented by the new predefined predicate \texttt{.is\_done}.
\item The body of the plan provides a lexical and runtime scope of the sub-plans, that is: 
variables used in the goal/context expressions are visible also to sub-plans
 the lifetime/availability of the sub-plans is limited to the time in which the g-plan is in execution
%
\item Failures generated by either the main sequence or by sub-plans generate a -!g that could handled.
%
\item a {\asl} plan \texttt{+!g : c <- b.} is an {\aser} g-plan with no g-plans.
\end{itemize}
%
%
%\subsection{Key points}

\noindent This extension turns out to bring a number of important benefits to
agent programming based on the BDI model, namely:
%
\begin{itemize}
\item improving the overall readability of the agent source code,
  reducing fragmentation and increasing modularity;
\item promoting a more goal-oriented programming style, enforcing yet
  preserving the possibility to specify purely reactive behaviour,
  properly encapsulated into plans for goals;
\item improving intention management, enforcing a one-to-one relation
  between intentions and goals --- so every intention is related to a
  (top-level) goal;
\item improving failure handling, in particular making it easier the
  management of failures related to plans for reacting to environment
  events;
\end{itemize}

%%%%%%%%%%%%%%%%%%%%%%%%%%%%%%%%%%%%%%%%%%%%%%%%%%%%%%%%%%%%%%%%%%%%%%%%%%%%%%%%%%%%%%
\section{Final Remarks}
\label{sec:conclusion}
%%%%%%%%%%%%%%%%%%%%%%%%%%%%%%%%%%%%%%%%%%%%%%%%%%%%%%%%%%%%%%%%%%%%%%%%%%%%%%%%%%%%%%

%
We formalised the main changes required in the existing formal semantics 
of AgentSpeak(L) and experimentally evaluated on first prototype
implementation of \aser, on top of the ASTRA and Jason platform
(available here\footnote{\url{https://github.com/agentspeakers}}).
%
Results will be described in a longer version of this paper.

%
As with any new programming language, there is much future work, 
including improving the prototype implementations and. comparing performances
%
More generally, full understanding and evaluation of a programming
language takes many years. We expect in the long term to use \aser\ in
the practical development of multi-agent systems, both for real-world
systems and also academic ones (e.g., for the multi-agent programming
contest~\cite{Albrecht18}). However, besides the actual programming
practice, we expect \aser\ to contribute to formal work as
well. Assessing how formal verification of \aser\ systems compares to
the original language is also planned as future work.

%As with any new programming language, there is much future work, 
%including improving the prototype implementations, comparing performances,
%and stressing the value of the new features in the context of
%real-world application case studies.

%%%%%%%%%%%%%%%%%%%%%%%%%%%%%%%%%%%%%%%%%%%%%%%%%%%%%%%%%%%%%%%%%%%%%%%%%%%%%%%%%%%%%%%%%%%%%%%%%%%%%%%%%
%% bibliography: see CFP for number of permitted pages

\bibliographystyle{ACM-Reference-Format}  % do not change this line!
\bibliography{main}  % put name of your .bib file here


\end{document}
