\section{Motivation}
\label{sec:motivation}

The main motivation behind {\aser} comes from the experience using
agent programming languages based on the {\asl} model, {\jason} and
ASTRA in particular.
%
Yet, these issues are relevant for any language based on the BDI
architecture.
%
The first issue is about the plan model.

% \subsection{Plan encapsulation}
%
In the BDI model, plans are meant to specify the means by which an
agent should satisfy an end~\cite{Rao96}.
%
In {\asl} a plan consists of a rule of the kind \textsf{e : c <- b}.
%
The head of a plan consists of a triggering event \textsf{e} and a context \textsf{c}.
%
The triggering event specifies why the plan was triggered, i.e., the addition or deletion of a belief or goal. 
%
The context specifies those beliefs that should hold in the agent's set of base beliefs, when the plan is triggered. 
%
The body of a plan is a sequence of actions or (sub-)goals.
%

%
% THE PROBLEM
%
So in this approach --- as well as in planning, in general --- the
\emph{means} to achieve a goal --- i.e., the body --- is meant to be
fully specified in terms of the actions the agent should execute and
the (sub-)goals the agent should achieve or test.
%
In the programming practice, however, it is often the case that the
strategy (the means) adopted to achieve some goal (the end) would
naturally include also \emph{reactions}, i.e., reacting to events
asynchronously perceived from the environment, including changes about
the beliefs.
% 
That is, an agent is not reacting only to unexpected events so as to
change/adapt its course of actions, but reactivity could be an
effective ingredient of the strategy, of the plan adopted to achieve
some goal.
%
Besides the programming practice, this is often the case if we
consider human activities too. Reactivity is a key ingredient of many
activities that we perform to achieve specific goals, not only to
handle events that represents errors or unexpected situations (for the
current courses of actions).
%
It follows naturally that this is also an opportunity to extend the
plan model so as to fully \emph{encapsulate} also reactions that are
part of the strategy to achieve the goal, as well as the subgoals that
are specific to that particular goal.


%
%
%
In the following, let's refer to plans triggered by event goals
% \texttt{+!g <- ...} 
as \emph{g-plans}, and plans triggered by the
belief change (including percepts) as \emph{e-plans}.
%
The BDI reasoning cycle automatically updates beliefs from percepts,
and this allows us to write down structured plans with courses of
actions that change according to the environment, by exploiting
goals/subgoals and contexts.
%
For instance, as a very simple example, yet capturing the point, let us consider the goal of  printing down all the numbers between N and 1, stopping if/when a 'stop' percept is perceived. 
%
This task can be effectively tackled using only g-plans:

\begin{small}
\begin{verbatim}
+!print_nums(0).
+!print_nums(_) : stop.
+!print_nums(N) : not stop <- println(N); !print_nums(N-1).
\end{verbatim}
\end{small}
%% RHB: should we replace println with .print ??
	
\noindent This is a clean solution, exploiting the context in plans
and the fact that in the reasoning cycle beliefs are automatically
updated given new percepts.
%
More generally, in the {\asl} model the suggested approach to realise
structured activities whose flow can be environment dependent is by
means of (sub-)goals and corresponding plans with a proper context.

In some cases however this approach is not fully effective, and
\emph{e-plans} are needed,
%
in particular every time we need to react \emph{while executing the
  body of a plan}.
%
A typical case occurs when we have actions (or subgoals) in a plan
that could take a long time to complete.
%
For instance, suppose that instead of simply printing we have a
long-term \texttt{elab} goal (it could be even an action):

\begin{small}
\begin{verbatim}
+!elab_nums(0).
+!elab_nums(N) : stop. 
+!elab_nums(N) : not stop <- !elab(N); !elab_nums(N-1).
\end{verbatim}
\end{small}

\noindent Suppose that, realistically, we cannot spread/pollute plans
about the goal \texttt{!elab} with a stop-dependent behaviour.
%
To solve this problem, using the {\asl} model and in BDI architectures
an e-plan can be introduced, using e.g. internal actions to act on the
current ongoing intention:

\begin{small}
\begin{verbatim}
+!elab_nums(0).
+!elab_nums(N) <- !elab(N); !print_nums(N-1).		
+stop <- .abort(elab_nums(_)).
\end{verbatim}
\end{small}
%% RHB: should we change .abort to .drop_intention ?

\noindent The problem here is that the e-plan \texttt{+stop <- ... } is
part of the strategy to achieve the goal \texttt{!elab\_nums}, however
it is encoded as a separate unrelated plan.

% \subsection{Explicit and Implicit Goals}

The use of e-plans to achieve goals is actually an important
conceptual brick of the {\asl} model.
%
Let us consider the robot cleaning example used to describe plans in
{\cite{Rao96}}.
%
One of the plans is:

\begin{small}
\begin{verbatim}
+location(waste,X):location(robot,X) & location(bin,Y)
  <- pick(waste); !location(robot,Y); drop(waste).
\end{verbatim}
\end{small}

\noindent That is, as soon as the robot perceives that there is a
waste in its location, then it can pick it up and bring it to the bin.
%
This plan an essential brick of the overall strategy to achieve the
goal of cleaning the environment --- or to maintain the environment
clean, in a \emph{maintenance goal} view.
%
However, the fact that this plan is useful to achieve that goal is not
explicit in the source code, it remains in the mind of the
programmer/designer.
%
%
%For instance, let's consider the classic toy example about the robot cleaning the environments from something (bombs, waste)~\cite{jason06}:
%
%\begin{small}
%\begin{verbatim}
%@p1
%+bomb(Terminal, Gate, BombType) : 
%  skill(BombType)
%  <- !go(Terminal, Gate);
%      disarm(BombType).
%@p2
%+bomb(Terminal, Gate, BombType) : 
%  ?skill(BombType)
%  <- !moveSafeArea(Terminal, Gate, BombType).
%@p3
%+bomb(Terminal, Gate, BombType) : 
%  not skill(BombType) & not ?skill(BombType)
%  <- .broadcast(tell, alter).
%...
%\end{verbatim}
%\end{small}
%
%This works very well, however it is not fully satisfactory from a conceptual point of view. 
%
The agent reacts to a waste in a location because it has an implicit
goal, which is about cleaning the environment.
%
However, since there are no g-plans about it, there is no explicit
trace in the agent mental structures about this goal.

%It is worth remarking that it is possible to bypass the problem in concrete languages (e.g. {\jason}), by exploiting specific features/tricks. 
%%
%However the solution is typically far from being satisfactory from a conceptual point of view.
%%
%For instance, about the bombing example - we can introduce an explicit  goal \texttt{!clean\_env} and a plan in which the course of action is simply waiting to perceive the environment is cleaned:
%
%\begin{verbatim}
%+!clean\_env <- .wait(all_bombs_dismantled).
%+bomb(Terminal, Gate, BombType) : skill(BombType)  <- ?
%\end{verbatim}
%
%\noindent On the one hand, this makes it possible to have an explicit intention at runtime about the dismantling bomb activity. However the plan p0 is completely unrelated in the code from the other reactive plans. Besides, when +bomb is perceived, a separate intention wrt the one about the dismantling goal is created. So they are treated as separate, not related activities.

% \subsection{Failure Handling}

This issue impacts also on plan failures handling, which is a very
important aspect of agent programming.
%
In the Jason dialect of {\asl}, we can define plans that handle the
failure of the execution of g-plans (generating the event
\texttt{-!g}), but not of e-plans.
%
So, for example, if there is a problem in the println action in:

\begin{small}
\begin{verbatim}
+stop <- println("stopped").
\end{verbatim}
\end{small}

\noindent the plan execution fails, without any possibility to react
and handle the failure.
%
In order to handle this, a programmer is forced to structure every
e-plan with failure handling using a subgoal:

\begin{small}
\begin{verbatim}
+stop <- !manage_stop.
+!manage_stop <- println("stopped").
-!manage_stop <- println("failure").
\end{verbatim}
\end{small}

\noindent This contributes to making the program longer and verbose,
besides increasing the number of plans to be managed by the
interpreter.

%Besides failure, meta-level actions to manage intentions (plans in execution) at runtime
%works only with g-plans, since they need to specify as argument the specific goal which generated the intention.

\bigskip

To summarise, this is the set of key issues identified for the basic
plan model in the practice of agent programming:
%
\begin{description}
%
\item[Lack of encapsulation:] The strategy to achieve a goal is
  fragmented among multiple plans, not explicitly related.
%
\item[Implicit vs. Explicit goals:] A part of the plans in a program
  have a goal which is implicit.
%
\item[Difficult failure handling:] Failures/errors generated in the
  body of e-plan cannot be directly captured.
%
\end{description}

%\noindent This has an impact also on the performance of the agent
%interpreter at runtime. In the traditional model, each reaction
%represented by an e-plan necessarily generates a new intention to
%manage it, increasing the number of plans in execution to be managed
%by the agent reasoning cycle.
%
%This could have also a negative impact on intention selection at
% each reasoning cycle~\cite{DBLP:conf/atal/LoganTY17}.

%\item[Difficult runtime management] -- Intentions generated by e-plans cannot be managed. Besides, as a consequence of point (1), it is not possible to manage (suspend, abort) the intentions related to some goals as a whole.
%
%\item[Runtime overhead] Each reaction (e-plans) is   new intention => new stack. 


\noindent Besides the pure programming level, it is worth noting that these
issues can affect the %whose
software engineering process concerning
agent development.
%%
In particular, at the design level, it is natural to specify
coarse-grained plans fully encapsulating the strategy to achieve or
maintain goals.
%
It would be important then to keep as much as possible the same level
of abstraction when going from the design to programming, and at
runtime too, to support agent reasoning.


%
%\begin{itemize}
%%
%\item Design / Engineering Drawbacks -- weak encapsulation => impact on modularity => impact on many aspects
%
%%
%\item Reasoning drawbacks -- proliferation of intentions => impact on the reasoning cycle, on intention selection
%%
%\end{itemize}


